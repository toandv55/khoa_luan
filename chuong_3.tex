\chapter{Công nghệ OSGi và kĩ nghệ phần mềm hướng thành phần}
Ở chương này công nghệ OSGi sẽ được giới thiệu và cách thức OSGi hỗ trợ trong kĩ nghệ phần mềm hướng thành phần.  
\section{Giới thiệu về OSGi}
Công nghệ OSGi - Open Service Gateway initiative, là mộ tập hợp các đặc tả định nghĩa một hệ thống thành phần cho nền tảng phát triển phần mềm Java. Những đặc tả này tạo lên một mô hình phát triển trong đó các hệ thống phần mềm được hợp thành một cách linh động từ các thành phần phần mềm có tính tái sử dụng. Các đặc tả của công nghệ OSGi cho phép các thành phần phần mềm che giấu đi các cài đặt bên trong mà chỉ giao tiếp với các thành phần khác thông qua các dịch vụ. Các dịch vụ này được chia sẻ một cách rõ ràng và xác định giữa các thành phần với nhau. Mô hình trên tuy đơn giản nhưng đã mang lại hiệu quả lớn trong hầu hết các khía cạnh trong quy trình phát triển phần mềm.

Mặc dù kĩ nghệ phần mềm hướng thành phần, cũng như các thành phần đã được đề xuất từ rất lâu, cuối những năm 1990 \cite{component-based-ian} tuy nhiên những ưu điểm của của hướng phát triển này vẫn chưa được áp dụng thành công nhiều trong việc phát triển phần mềm. Công nghệ OSGi là công nghệ đầu tiên thực sự thành công với với việc xây dựng các hệ thống hướng thành phần mà giải quyết được rất nhiều vấn đề thực tế trong phát triển phần mềm. Áp dụng công nghệ OSGi giúp việc giảm đi khá nhiều độ phức tạp trong quá trình phát triển. Mã nguồn dễ viết hơn, dễ kiểm thử hơn, tính tái sử dụng được nâng cao, xây dựng hệ thống trở lên đơn giản hóa hơn rất nhiều, việc triển khai các hệ thông trở lên dễ kiểm soát hơn, các lỗi dễ phát hiện sớm, môi trường thực thi cung cấp rất nhiều sự quan sát tới việc các thành phần nào đang hoạt động. Công nghệ OSGi đã được sử dụng, thẩm định và chấp nhận trong nhiều hệ thống phần mềm phổ biến như Eclipse và Spring.\cite{whatisosgi}

Phiên bản đặc tả đầu tiên của công nghệ OSGi được phát hành vào năm 2000, phiên bản đặc tả hiện tại này là 5.0.0. Từ phiên bản 4.2 phát hành năm 2010, OSGi đã có thêm một phiên bản thương mại, Enterprise OSGi, phục vụ cho việc xây dựng các hệ thống thương mại bằng cách tích hợp các công nghệ của nền tảng Java Enterprise Edition (JEE) bên cạnh phiên bản cơ bản là Core OSGi \cite{specs}.

Nói một ngắn gọn, công nghệ OSGi cung cấp một mô hình phát triển hiện thực hóa được các ưu điểm của mô hình phát triển hướng thành phần.

\section{Mô hình nhiều lớp của OSGi}

OSGi có một mô hình nhiều lớp \cite[p.~215]{whatisosgi} được thể hiện trong hình \ref{fig:layers}
\begin{figure}[htbp]
	\centering
		\includegraphics{image/layers.JPG}
	\caption{Mô hình nhiều lớp của OSGi}
	\label{fig:layers}
\end{figure}


Dưới đây là mô tả sơ lược cho các lớp trong hình \ref{fig:layers}
\begin{itemize}
	\item Bundles - Là thành phần phần trong OSGi được phát triển với lập trình viên, là thành phần tạo lên các hệ thống phần mềm dựa trên OSGi, một thành phần gọi là một bundle.
	\item Services - Lớp có nhiệm vụ kết nối các thành phần một cách linh động, tìm và gắn kết các dịch vụ (là các đối tượng trong ngôn ngữ lập trình Java),  giúp các thành phần có thể giao tiếp và cộng tác với nhau.
	\item Life-Cycle - Là lớp cung cấp cách thức quản lý các thành phần như việc thêm mới một thành phần, khởi động, dừng hoạt động, cập nhật hay gỡ bỏ.
	\item Modules - Là lớp định nghĩa cách thức các thức một bundle sử dụng mã nguồn từ bundle khác cũng như cách đưa mã nguồn ra bên ngoài để được sử dụng bởi các bundle khác. 	
	\item  Security - Là lớp xử lý các vấn đề liên quan tới bảo mật.
	\item  Execution Environment - Cung cấp môi trường thực thi cho các thành phần có thể hoạt động.
\end{itemize}

\section{OSGi hỗ trợ kĩ nghệ phần mềm hướng thành phần}
Ở trong mục 2.1 của chương 2 các thành phần cơ bản của kĩ nghệ phần mềm hướng thành phần bao gồm :
\begin{enumerate}
	\item Các thành phần phần mềm
	\item Các chuẩn thành phần
	\item Các thành phần trung gian - Middleware
	\item Một quy trình phát triển
\end{enumerate}
Từ đây ta sẽ chỉ ra cách mà công nghệ OSGi hỗ trợ kĩ nghệ phần mềm hướng thành phần.
\subsection{Thành phần phần mềm của OSGi}
Trong kĩ nghệ phần hướng thành phần thì các thành phần phần mềm (software components) là thành phần cơ bản và trọng tâm. Trong mục này cách các thành phần phần mềm được triển khai trong công nghệ OSGi sẽ được trình bày.
Các thành phần phần mềm trong OSGi là các OSGi bundles, trước khi chứng minh các OSGi bundles có đầy đủ các đặc điểm cơ bản của các thành phần phần mềm, ta sẽ tìm hiểu về các đặc điểm cơ bản cũng như cấu trúc của các OSGi bundles.
\subsection{Các OSGi bundles \cite{osgicore5}}
OSGi là một công nghệ dành cho việc phát triển các hệ thống phần mềm hướng thành phần cho nền tảng Java vì thế các OSGi bundles chính là các file JAR \cite{jar}. Về cơ bản, các OSGi bundles có các thành phần không khác thì các file JAR thông thường bao gồm : các files .class (Java byte code), các file tài nguyên như .xml, .jpg và một file chứa siêu dữ liệu (thông tin về chính file JAR) Manifest.mf. Hình \ref{fig:bundle} là cấu trúc của một OSGi bundle.
\begin{figure}[htbp]
	\centering
		\includegraphics{image/bundle.JPG}
	\caption{Cấu trúc của một OSGi bundle}
	\label{fig:bundle}
\end{figure}
 
Tuy nhiên điểm khác biệt lớn nhất giữa OSGi bundles và các file JAR thông thường nằm ở nội dung của file siêu dữ liệu Manifest.mf. Trong khi file Manifest.mf của một file JAR thông thường chỉ chứa một số thông tin cơ bản về file JAR đó thì file Manifest.mf của các OSGi bundles chứa thêm nhiều trường thông tin quan trọng khác và sẽ được đọc và xử lý ở tầng module của OSGi khi các bundle được triển khai vào môi trường thực thi. Các thông tin trong một file Manifest.mf của một OSGi bundle sẽ chứa các trường dữ liệu cơ bản như sau, trong đó một số trường là bắt buộc phải có.
\begin{itemize}
\item Bundle-SymbolicName: <tên của bundle> - Chứa tên của một bundle, sử dụng cùng mới phiên bản của bundle tao lên một định danh duy nhất cho một bunlde, trường dữ liệu này là bắt buộc phải có.
\item Bundle-Version: <số phiên bản của bundle> - Cùng với tên bundle tạo lên định danh duy nhất cho bundle, trường này cũng là bắt buộc.
\item Export-Package: <tên các java packages> - Chứa tên các Java packages mà các bundles khác có thể sử dụng được, các packages được liệt kê mặc định là chỉ được nhìn thấy và sử dụng bên trong chính bundle, nếu trường này không được khai báo thì tất cả các Java packages là bị che giấu và không thể được sử dụng ở bên ngoài phạm vi bundle. Các Java packages này khi luôn được gắn kèm với số phiên bản của bundle.
\item Import-Package: <tên các java packages> - Chứa tên của các Java packages mà được chia sẻ bởi bởi các bundles khác (được khai báo trong trường Export-Package của các bundles khác), tên của các Java packages này thường đi kèm với số phiên bản của bundle mà chúng nằm trong. Nếu các Java packages từ các bundles khác được sử dụng mà không được khai báo rõ trong trường này thì khi triển khai trong môi trường thực thi thì bundle sẽ không thể chuyển trong trạng thái sẵn sàng hoạt động. Và sự thiếu xót của các Java packages này sẽ được thông báo một cách tường minh khi bundle được triển khai trong môi trường thực thi.
\item Export-Service: <tên các java interfaces> - Chứa tên của các Java interfaces mà thông qua đó các dịch vụ mà bundle cung cấp sẽ có thể được sử dụng ở bên ngoài, từ các bundles khác.
\item Import-Service: <tên các java interfaces> - Chứa tên của các Java interfaces ở trong các bundles khác, chứa trong các Java packages được khai báo trong tường Export-Package, thông qua các interfaces này các dịch vụ được cung cấp từ các bundles khác được sử dụng bên trong một bundle. Có nghĩ các dịch vụ này phải được cung cấp thì bundle mới có thể hoạt động đúng đắn.
\end{itemize}  
Bên dưới là nội dung một file Manifest.mf lấy từ một bundle trong hệ thống hỗ trợ học viên cao học đăng kí đề tài luận văn tại khoa công nghệ thông tin, Đại học Công nghệ.
\begin{lstlisting}[label=manifest, 
inputencoding=utf8,
breaklines=true,
basicstyle=\ttfamily\footnotesize,
caption=Nội dung của môt file manifest của OSGi bundle]

Manifest-Version: 1.0
Build-Jdk: 1.7.0_51
Built-By: dotoan
Bundle-ManifestVersion: 2
Bundle-Name: postgraduate
Bundle-SymbolicName: postgraduatems-postgraduate
Bundle-Version: 1.0.0
Created-By: Apache Maven Bundle Plugin
Export-Service: postgraduatems.postgraduate.services.EditPostgraduateBO, postgraduatems.postgraduate.services.PostgraduateBO, postgraduatems.postgraduate.services.ThesisUpdateBO
Import-Service: postgraduatems.persistence.services.LecturerDAO;multiple:false, postgraduatems.persistence.services.PostgraduateDAO;multiple:=false
Import-Package: postgraduatems.persistence.entities;version="[1.0,2)", postgraduatems.persistence.services;version="[1.0,2)", postgraduatems.postgraduate.services;version="[1.0,2)"
\end{lstlisting}


Bundle trên có tên là "postgraduatems-postgraduate", số phiên bản là 1.0.0, bundle 
này cung cấp ba dịch vụ thông qua ba Java interfaces là \seqsplit{postgraduate.services.EditPostgraduateBO}, 
\seqsplit{postgraduatems.postgraduate.services.PostgraduateBO} và 
\seqsplit{postgraduatems.postgraduate.services.ThesisUpdateBO}.
Các dịch vụ từ các bundle khác được sử dụng dụng trong bundle này thông qua các Java interfaces là \seqsplit{postgraduatems.persistence.services.LecturerDAO} và 
\seqsplit{postgraduatems.persistence.services.PostgraduateDAO}, 
đây là các dịch vụ liên quan tới việc giao tiếp cơ sở dữ liệu.
Bundle này chỉ sử dụng ba Java package từ bundle bên ngoài là postgraduatems.persistence.entities, \seqsplit{postgraduatems.persistence.services} và \seqsplit{postgraduatems.postgraduate.services} trong với phiên bản thấp nhất là 1.0 và phiên bản cao nhất không vượt quá 2. Bundle "postgraduatems-postgraduate" với số phiên bản là 1.0.0 không đưa ra bên ngoài một Java package nào cả, điều này có nghĩa là cài đặt bên trong của bundle này hoàn toàn được che giấu với bên ngoài, cách duy nhất để giao tiếp với bundle này là thông quan giao diện được thể hiện bởi các Java interfaces mà bundle này đăng kí sẽ cung cấp dịch vụ thông qua. Ở đây thậm chí các Java interfaces bundle này sử dụng để cung cấp dịch vụ cũng được lấy từ bên ngoài, tức là mã nguồn bên trong bundle này không hề được biết đến, nó hoàn toàn đóng. 

\subsubsection{Các file JAR thông thường không thể được dùng như các thành phần}
Các file JAR là một cách đóng gói, lưu trữ các file .class và các tài nguyên khác \cite{jar} với mục đích để sử dụng trong các ứng dụng viết bằng Java thông thường được dùng như các thư viện hỗ trợ. Tuy nhiên cách đóng gói các files .class thành một file JAR chỉ mang ý nghĩa về mặt vật lý, tức là chỉ giúp phân tách, tổ chức các files .class vào các đơn vị lưu trữ khác nhau giúp cho việc quản lý và sử dụng dễ dàng hơn. Tuy nhiên các file JAR này không hề có cơ chế để kiếm soát việc sử dụng các Java class bên trong chúng. Một khi một file JAR được đưa vào trong classpath (chứa đường dẫn tới các thư viện) của một file JAR khác thì tất cả các Java class của file JAR được đưa vào sẽ luôn được nhìn thấy và sử dụng bởi trong file JAR chứa đường dẫn tới file JAR mà các Java class này nằm trong. Bản thân ngôn ngữ lập trình Java chỉ có cơ chế kiểm soát việc sử dụng các Java class ở một mức thấp hơn đó là package, trong khi một file JAR chứa rất nhiều packages cho nên việc áp dụng cơ chế kiểm soát các Java class ở mức package không mang lại hiệu quả để các file JAR thông thường trở thành các thành phần, và khi được tải vào máy ảo Java thì ranh giới của các file JAR biến mất, và lúc này không thể biết được Java class nào thuộc về file JAR nào. Trong khi việc kiểm soát, che giấu cài đặt bên trong là đặc điểm cơ bản của một thành phần phần mềm tái sử dụng, cho nên một file JAR thông thường không thể được sử dụng như một thành phần mềm, và càng không thể dùng để xây dựng các hệ thống hướng thành phần thực sự.

\subsection{Các dịch vụ của OSGi bundle}
Mỗi bundle trong OSGi sẽ cung cấp một số các dịch vụ, đồng thời mỗi bundle cũng có thể sử dụng dịch vụ từ các bundle khác. Nói một cách khác, các OSGi bundle giao tiếp và làm việc với nhau thông qua các dịch vụ. Mỗi dịch vụ trong OSGi bundle gồm hai thành phần :
\begin{enumerate}
\item Giao diện dịch vụ (service interfaces) - được dùng để đăng kí dịch vụ cũng như dùng để tìm kiếm và sử dụng dịch vụ, các giao diện dịch vụ chính là các Java interfaces, giống như những bản hợp đồng, thỏa thuận giữa hai bên, bên cung cấp và sử dụng dịch vụ. Bên cung cấp sẽ cung cấp dịch vụ thông qua các giao diện này, và bên sử dụng chỉ cần biết các giao diện này để có thể sử dụng được dịch vụ mà không cần biết dịch vụ được cài đặt như thế nào.
\item Cài đặt của dịch vụ (service implementation) - Cài đặt của dịch vụ thực chất là các Java class triển khai các Java interfaces mà dịch vụ sẽ được cung cấp thông qua. Tạo một dịch vụ chính là việc tạo một đối tượng của Java class mà chứa cài đặt của dịch vụ và đăng kí đối tượng này là một dịch vụ thông qua một Java interface. Việc sử dụng dịch vụ trình là việc sử dụng các phương thức có trong đối tượng được đăng kí, các phương thức này phải được quy định trong Java interface.
\end{enumerate} 

Trong OSGi thì lớp Service sẽ chịu trách nhiệm cho việc đăng kí các dịch vụ thông qua các giao diện dịch vụ của các bundle cũng như việc tìm kiếm các dịch vụ hay các đối tượng Java và kết nối dịch vụ với bundle yêu cầu dịch vụ để được sử dụng.
Hình \ref{fig:service_reg} mô tả quá trình đăng kí và sử dụng dịch vụ của OSGi bundles.
\begin{figure}[htbp]
	\centering
		\includegraphics{image/service_reg.JPG}
	\caption{Quá trình đăng kí và sử dụng dịch vụ của OSGi bundles}
	\label{fig:service_reg}
\end{figure}

Quá trình đăng kí diễn ra như sau 
\begin{enumerate}
\item Bundle cung cấp dịch vụ đăng kí cung cấp một dịch vụ, được cài đặt bởi một đối tượng Java, thông qua một Java interface (giao diện dịch vụ) với lớp Service của OSGi.
\item Bundle cung cấp dịch vụ công khai Java interface (giao diện dịch vụ) cho các bundle sử dụng dịch vụ.
\item Bundle sử dụng dịch vụ dùng Java interface đã được công khai, yêu cầu dịch vụ tới lớp Service
\item Lớp Service dùng Java interface để tìm kiếm dịch vụ tương ứng, và kết nối dịch vụ (đối tượng Java) với bundle sử dụng.
\item Bundle sử dụng dịch vụ của bundle cung cấp, tuy nhiên bundle sử dụng không biết dịch vụ thực sự chạy ở đâu mà sự liên kết này là hoàn toàn động.
\end{enumerate} 

\subsection{Triển khai các OSGi bundle}
OSGi cung cấp một môi trường thực thi mà qua đó các OSGi bundle sẽ được triển khai và và được chạy. Hình \ref{fig:runtime} thể hiện một môi trường thực thi của các OSGi bundle.
\begin{figure}[htbp]
	\centering
		\includegraphics{image/runtime.JPG}
	\caption{Môi trường thực thi của các OSGi bundle}
	\label{fig:runtime}
\end{figure}
Bên cạnh đó việc triển khai và quản lý các bundle cũng được thực hiện một cách linh động và thuận tiện bởi lớp Life-Cycle của OSGi. Hình \ref{fig:states} mô tả các trạng thái của một OSGi bundle và các thao tác để chuyển trạng thái của bundle.
\begin{figure}[hbtp]
	\centering
		\includegraphics{image/states.JPG}
	\caption{Các trạng thái của một OSGi bundle}
	\label{fig:states}
\end{figure}

Các trạng thái của một 

\subsection{OSGi bundles là các thành phần phần mềm}
Ở trên ta đã thấy tại sao một file JAR thông thường không thể 

 How does OSGI support component-based... and modularity.
advantages over pure Java
 Types(?) of OSGIs
- Descriptive
- Spring OSGI
- Blueprint
for each (description, sample code, pros and cons,)

conclude (what's the best for now, why)
