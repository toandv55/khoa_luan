\chapter{Kết luận}
Trong quá trình thực hiện khóa luận này tôi đã thu được rất nhiều kiến thức bao gồm cả lý thuyết và thực hành.

Về kĩ nghệ phần mềm hướng thành phần, tôi đã nắm các đặc điểm cơ bản của kĩ nghệ phần mềm hướng thành phần, thế nào là các thành phần phần mềm hướng thành phần, các chuẩn của một mô hình phát triển hướng thành phần và các lợi ích của việc phát triển một hệ thống theo hướng thành phần.\\

Tôi cũng đã áp dụng các lý thuyết của kĩ nghệ phần mềm thành phần để chứng tỏ được một công nghệ là OSGi đáp ứng đầy đủ các đặc điểm cơ bản của một mô hình phát triển phần mềm hướng thành phần. Bên cạnh đó tôi cũng tìm hiểu và sử dụng được công nghệ OSGi áp dụng xây dựng một hệ thống hướng thành phần. Nhưng hơn hết trong quá trình xây dựng và phát triển hệ thống hỗ trợ học viên cao học đăng kí luận văn, tôi đã học được rất nhiều các kiến thức và kĩ năng thực tiễn như việc thiết kế một hệ thống, vận dụng được các nguyên lý phân tích thiết kế hướng đối tượng, các kĩ thuật thiết kế hệ thống, thiết kế cơ sở dữ liệu. Bên cạnh đó tôi còn có cơ hội tiếp cận và sử dụng các công nghệ mới như JPA, cách biên dịch và đóng gói tự động sử dụng Apache Maven hay cách quản lý mã nguồn, quản lý phiên bản phần mềm sử dụng Git. \\

OSGi là một công nghệ mới khi đang được tích hợp với nền tảng Java Enterprise để xây dụng các hệ thống lớn, hướng thành phần và thực sự thể hiện được rất nhiều ưu điểm và có nhiều tiềm năng. OSGi với bản đặc tả Enterprise còn chứa rất nhiều các thành phần hỗ trợ xây dựng một hệ thống thành phần và có thể phân tán trên nhiều nơi. Tuy nhiên OSGi vẫn có một số điểm hạn chế như là một công nghệ chưa được phổ biến, tài liệu liên quan chưa nhiều, và các công nghệ hỗ chưa đầy đủ. Trong dự định sắp tới tôi sẽ nghiên cứu thêm về công nghệ OSGi bên cạnh đó cũng có sự tìm hiểu với các mô hình hướng thành phần như Web Service hay Java Enterprise Beans để tìm ra các điểm mạnh và yếu của từng mô hình để có thể áp dụng vào phát triển các hệ thống thực tế sau này.