\chapter{Giới thiệu}


Trong ngành phát triển phần mềm hiện nay, các hệ thống phần mềm càng ngày càng có xu hướng lớn hơn về quy mô và gia tăng về độ phức tạp, ví dụ hệ thống phần mềm thương mại dành cho doanh nghiệp, trong khi nhu cầu cho các hệ thống này ngày càng tăng. Việc chuyển giao và triển khai các hệ thống lớn và phức tạp một cách nhanh chóng là việc không đơn giản. Để giải quyết vấn đề thời gian chuyển giao và triển khai cũng như độ phức tạp và quy mô của các hệ thống này thì cách tốt nhất là tái sử dụng các thành phần phần mềm có sẵn thay vì cài đặt mới hoàn toàn các hệ thống mới.
Phát triển phần mềm hướng thành phần là một cách tiếp cận phát triển các hệ thống phần mềm dựa trên việc tái sử dụng lại các thành phần phần mềm.
Các thành phần phần mềm (software components) ở mức trừu tượng hóa cao và được xác định bởi các giao diện (interfaces). Phát triển phần mềm theo cách tiếp cận hướng thành phần là quá trình định nghĩa, cài đặt, tích hợp hay lắp ghép các thành phần phần mềm độc lập tạo lên các hệ thống phần mềm hoàn chỉnh.

Trong khóa luận này, tôi sẽ trình bày về kĩ nghệ phần mềm hướng thành phần dựa trên cơ sở lý thuyết và cũng như sử dụng một công nghệ hỗ trợ việc phát triển các hệ thống phần mềm hướng thành phần - Công nghệ OSGi. Bên cạnh đó, một hệ thống phần mềm thực nghiệm cũng được phát triển sử dụng công nghệ OSGi theo hướng tiếp cận hướng thành phần - Hệ thống hỗ trợ đăng kí luận văn cho học viên cao học tại khoa Công nghệ thông tin, Đại học Công nghệ, Đại học Quốc gia Hà Nội.
