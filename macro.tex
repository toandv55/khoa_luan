
%=============================================================================================================================
% NEW THEOREMS, ENVIRONMENTS, COMMANDS

\newtheorem{definition}{Định nghĩa}
\newtheorem{lemma}{Bổ đề}
\newtheorem{theorem}{Định lý}

%
\def \bp{khoảng bận rộn mức-$i$}
\def \bph{khoảng bận rộn xấp xỉ mức-$i$}
\def \fea{thực hiện được}
\def \wcrt {$wcrt$}
\def \at{test xấp xỉ}
\def \ss{hệ thống với độ ưu tiên cố định}
\def\cd{kỳ hạn ràng buộc}
\def\ad{kỳ hạn không ràng buộc}
\def \fp{độ ưu tiên cố định}

% 15-05-13
\def \ap{khoảng hoạt động mức-$i$}
\def \apj{khoảng hoạt động mức-$(i,l)$}
\def \aph{khoảng hoạt động xấp xỉ mức-$i$}
\def \apjh{khoảng hoạt động xấp xỉ mức-$(i,l)$}

%end
%22-05
\def \pi {khoảng nguyên tố}
\def \rf {hàm đáp ứng}
\def \of {hàm chiếm dụng}


%=============================================================================================================================
%
\newcommand{\drop}[1]{} % Comment multi-line
\newcommand{\pr}[1]{{\color{black} #1}} 
\newcommand{\clast}[1]{{\color{black} #1}}
\newcommand{\cclast}[1]{{\color{black} #1}}
%\newcommand{\nfstrike}[1]{{\color{black}\sout{#1}}} can be blue yellow green red
\newcommand{\chau}[1]{{\color{black} #1}}
\newcommand{\chaumoi}[1]{{\color{black} #1}}
\newcommand{\ccuoi}[1]{{\color{black} #1}}
\newcommand{\chaunua}[1]{{\color{black} #1}}
\newcommand {\khue} [1]{{\color{black} #1}}
\newcommand {\thanh} [1]{{\color{black} #1}}
\newcommand {\abc} [1]{{\color{yellow} #1}}
%============================================================================

\newcommand{\FP}{{\sc fp}}
\newcommand{\DM}{{\sc dm}}
\newcommand{\RM}{{\sc rm}}
\newcommand{\RMA}{{\sc rta}}
\newcommand{\PDA}{{\sc pda}}
\newcommand{\RTA}{{\sc rta}}
\newcommand{\edf}{\textsc{edf}}
\newcommand{\WCET}{{\sc wcet}}
\newcommand{\WCOT}{{\sc wcot}}
\newcommand{\WF}{{\sc wf}}
\newcommand{\AS}{{\sc as}}
\newcommand{\NP}{{\sc np}}
\newcommand{\PTAS}{{\sc ptas}}
\newcommand{\FPTAS}{{\sc fptas}}
\newcommand{\bigOh}{\mathcal{O}}
\newcommand{\bigO}{\mathcal{O}}

%============================================================================
%marking: pour rapport, il faut définir, pour article, il ne faut pas. 
\drop{\newcommand{\bdiv}{\mbox{~div~}}
\newcommand{\qed}{\rule{7pt}{7pt}}}

% "box" symbols at end of s
\def\QEDclosed{\mbox{\rule[0pt]{1.3ex}{1.3ex}}} % for a filled box
% V1.6 some journals use an open box instead that will just fit around a closed one
\def\QEDopen{{\setlength{\fboxsep}{0pt}\setlength{\fboxrule}{0.2pt}\fbox{\rule[0pt]{0pt}{1.3ex}\rule[0pt]{1.3ex}{0pt}}}}
\def\QED{\QEDclosed} % default to closed
\def\proof{\noindent{\itshape Chứng minh: }}
\def\endproof{\hspace*{\fill}~\QED\par\endtrivlist\unskip}

%======================================================================================

\newenvironment{exam}{\begin{exa} \rm}{\hspace*{\fill}$\qed$\end{exa}}
\newenvironment{Example}{\begin{exa} \rm}{\hspace*{\fill}$\qed$\end{exa}}
\newenvironment{Proof}{\noindent{\bf Chứng minh : }}{\qed}
\newenvironment{Pro}{\noindent{ \textit{Chứng minh.} }}{\qed}%Chau 18-04 bo \bf, them \textit, bo :, them .
\newenvironment{LabelledProof}[1]{\noindent{\bf Proof of #1}}{}%{\qed}
\newenvironment{LabelledSketch}[1]{\noindent{\bf Sketch of proof of #1: }}{}%{\qed}
\newenvironment{Sketch}{\noindent{\bf Sketch of proof: }}{}%{\qed}
\newcommand{\beginProof}[1]{\noindent{\bf Chứng minh:}}
\newcommand{\finishProof}[1]%
{~\newline \noindent{\bf \qed\   End of Proof} (de {\bf #1}).\newline}
\newcommand{\ProofSpace}{\medskip}

%========================================================================================
\newtheorem{Def}{Định nghĩa} % Dinh nghia
\newtheorem{Coro}{Hệ quả} % He qua
\newtheorem{Th}{Định lý} % Dinh ly
\newtheorem{Le}{Bổ đề} % Bo de
\newtheorem{Prop}{Property}
\newtheorem{exa}{Ví dụ}
\newtheorem{Rmk}{Remark}
\newtheorem{proo}{Chứng minh} %Chung minh


%====================================================================================
\def\sec{\section}
\def\ssec{\subsection}
\def\sssec{\subsubsection}
\def\ssssec{\subsubsubsection}
\def\para{\paragraph}
\def\spara{\subparagraph}

%18-04 Chau sua lai het cac beg
\def\begdef{\begin{definition}}
\def\enddef{\end{definition}}
\def\begth{\begin{theorem}}
\def\endth{\end{theorem}}
\def\begle{\begin{lemma}}
\def\endle{\end{lemma}}
\def\begco{\begin{corollary}}
\def\endco{\end{corollary}}
\def\begpreu{\begin{proof}}
\def\begpro{\begin{Pro}}
\def\endpreu{\end{proof}}
\def\endpro{\end{Pro}}
\def\begprop{\begin{Prop} }
\def\endprop{\end{Prop}}
\def\begex{\begin{example}}
\def\endex{\end{example}}
\def\begrmq{\begin{Rmk}}
\def\endrmq{\end{Rmk}}
\def\begrmk{\begin{Rmk}}
\def\endrmk{\end{Rmk}}
\def\begdes{\begin{description} \item[] }
\def\enddes{\end{description}}
\def\begsket{\begin{Sketch}  }
\def\endsket{\end{Sketch}}

\def\begeq{\begin{equation}  }
\def\endeq{\end{equation}}
\def\begeqno{\begin{equation*}  }
\def\endeqno{\end{equation*}}
\def\begarr{\begin{eqnarray}}
\def\endarr {\end{eqnarray}}
\def\begarrno{\begin{eqnarray*}}
\def\endarrno {\end{eqnarray*}}
\def\begsubeq{\begin{subequations}  }
\def\endsubeq{\end{subequations}}
\def\begsubeqno{\begin{subequations*}  }
\def\endsubeqno{\end{subequations*}}
%\def\bega{\begin{gather}}
%\def\endga {\end{gather}}

%===============================================================================
%18-04 Chau sua lai tat refence rut gon
\newcommand {\rchap}[1]{\ref{#1}}
\newcommand {\rdef}[1]{ \ref{#1}}
\newcommand {\rth}[1]{ \ref{#1}}
\newcommand {\rle}[1]{\ref{#1}}
\newcommand {\rco}[1]{Corollary \ref{#1}}
\newcommand {\req}[1]{(\ref{#1})}
\newcommand {\rsec}[1]{Sect. \ref{#1}}
\newcommand {\rssec}[1]{Subsect. \ref{#1}}
\newcommand {\rsssec}[1]{Subsubsect. \ref{#1}}
\newcommand {\rfig}[1]{\ref{#1}}
\newcommand {\rtab}[1]{Table \ref{#1}}
\newcommand {\rex}[1]{Example \ref{#1}}

%=================================================================================
\newcommand{\equals}{\stackrel{\operatorname{def}}{=}}
\newcommand{\sig}[1]{\sum_{#1}}
\newcommand{\flfrac}[2]{\left\lfloor\frac{#1}{#2}\right\rfloor}
\newcommand{\cefrac}[2]{\left\lceil\frac{#1}{#2}\right\rceil}
\newcommand{\flfr}[2]{\left\lfloor\frac{#1}{#2}\right\rfloor}
\newcommand{\cefr}[2]{\left\lceil\frac{#1}{#2}\right\rceil}
\newcommand{\ce}[1]{\left\lceil #1 \right\rceil}
\newcommand{\fl}[1]{\left\lfloor #1 \right\rfloor}
\newcommand{\lef}{\left\{}% pre-defined
\newcommand{\ri}{\right\}}
\newcommand{\exc}{\backslash}
\newcommand{\infer}{\Longrightarrow}
\newcommand{\equi}{\Longleftrightarrow}
\newcommand{\ud}{\mathrm{d}}

\newcommand{\limup}[1]{\lim_{t\uparrow #1}}
\newcommand{\limdown}[1]{\lim_{t\downarrow #1}}
%=============================================================================

\def\Ti{{\textsc{{T}}}_{i}}% 20-04 Chau tu sc thanh mbox
\def\Ttt{\textsc{{H}}}%_{t_1, t_2}} %14-03 Chau 20-04 Chau tu sc thanh mbox

\def\t{\tau_i}
\newcommand{\tp}[1]{\tau_{#1}}
\newcommand{\job}{\tau_{i,l}}
\newcommand{\jp}[1]{\tau_{i,#1}}
\newcommand{\jpp}[2]{\tau_{#1,#2}}   

%\def\r{r_i}
%\def\ri{r_i}
\def\C{C_i}
\def\D{D_i}
%\def\d{d_i}
\def\T{T_i}
\def\U{U_i}
\def\J{J_i} 
\def\F{F_i}
\def\B{B_i}

\def\Cp{C_{i,p}}

\def\Fn{F_n}
\def\Bn{B_n}

\def\m{m_i}


%==========================================================================

\newcommand{\rbf}{\text{\textsc{rbf}}(\tau_i, t)} 
\newcommand{\rbfp}[1]{\text{\textsc{rbf}}(\tau_{#1}, t)}

\newcommand{\rbfP}{\dddot{\text{\textsc{rbf}}} (\tau_i, t)}
\newcommand{\rbfPp}[1]{\dddot{\text{\textsc{rbf}}} (\tau_{#1}, t)}

\newcommand{\rbftp}[1]{\wildetilde{\text{\textsc{rbf}}}(\tau_{#1},t)}


% 15-05-13
\newcommand{\rbff}{\text{\textsc{rbf}}'(\tau_i, t)} 
\newcommand{\rbffp}[1]{\text{\textsc{rbf}}'(\tau_{#1}, t)} 

\newcommand{\rbfh}{\widehat{\text{\textsc{rbf}}}(\tau_i, t)}
\newcommand{\rbfhp}[1]{\widehat{\text{\textsc{rbf}}}(\tau_{#1}, t)}

\newcommand{\rbffh}{\widehat{\text{\textsc{rbf}}'}(\tau_i, t)} 
\newcommand{\rbffhp}[1]{\widehat{\text{\textsc{rbf}}'}(\tau_{#1}, t)}


%======================================================================================

\newcommand{\w}{W_{i, l}(t)}
\newcommand{\wh}{\widehat{W}_{i, l}(t)}
\newcommand{\wt}{\widetilde{W}_{i, l}(t)}
% Thuong modify - 1/3/2013
\newcommand{\whi}{\widehat{W}_{i}(t)}
\newcommand{\wi}{W_{i}(t)}

\newcommand{\wj}[1]{W_{i, #1}(t)}         %\wp is alreadly defined
\newcommand{\whp}[1]{\widehat{W}_{i, #1}(t)}
\newcommand{\wtp}[1]{\widetilde{W}_{i, #1}(t)}

\newcommand{\wpp}[2]{W_{#1, #2}(t)}
\newcommand{\whpp}[2]{\widehat{W}_{#1, #2}(t)}
\newcommand{\wtpp}[2]{\widetilde{W}_{#1, #2}(t)}

\newcommand{\W}{W_i(t)}
\newcommand{\Wh}{\widehat{W}_i(t)}
\newcommand{\Wt}{\widetilde{W}_i(t)}

\newcommand{\Wp}[1]{W_{#1}(t)}
\newcommand{\Whp}[1]{\widehat{W}_{ #1}(t)}
\newcommand{\Wtp}[1]{\widetilde{W}_{ #1}(t)}

\newcommand{\s}{S_i}
\newcommand{\sh}{\widehat{S}_i}
\newcommand{\st}{\widetilde{S}_i}

%======================================================================================
\drop
{
\newcommand{\la}{\text{\textsc{la}}^{BB}(\tau_i, t)}
\newcommand{\lap}[1]{\text{\textsc{la}}^{BB}(\tau_{#1}, t)}

\newcommand{\lan}[1]{\text{\textsc{la}}_{#1}(\tau_i, t)}
\newcommand{\lanp}[2]{\text{\textsc{la}}_{#1}(\tau_{#2}, t)}

%Bo di cung duoc: chac ko? Ko, chi 3 cai cuoi (trong article  ni) duoc 
\newcommand{\LA}{\text{\textsc{la}}^{BB}(\tau_i, t)}
\newcommand{\LAp}[1]{\text{\textsc{la}}^{BB}(\tau_{#1}, t)}
\newcommand{\LAn}[1]{\text{\textsc{la}}_{#1}(\tau_i, t)}
\newcommand{\LAnp}[2]{\text{\textsc{la}}_{#1}(\tau_{#2}, t)}

\newcommand{\LA}{\text{\textsc{la}}(\tau_i, t)}
\newcommand{\LAp}[1]{\text{\textsc{la}}(\tau_{#1}, t)}
}

\newcommand{\delt}{\delta(\tau_i, t)}
\newcommand{\lamb}{\lambda(\tau_i, t)}
\newcommand{\gamm}{\gamma(\tau_i, t)}
\newcommand{\delp}[1]{\delta(\tau_{#1}, t)}
\newcommand{\lamp}[1]{\lambda(\tau_{#1}, t)}
\newcommand{\gamp}[1]{\gamma(\tau_{#1}, t)}

\newcommand{\RBF}{\text{\textsc{rbf}}} 
\newcommand{\RBFp}{\dddot{\text{\textsc{rbf}}}}
\newcommand{\RBFh}{\widehat{\text{\textsc{rbf}}}}
\newcommand{\RBFt}{\wildetilde{\text{\textsc{rbf}}}}
\newcommand{\LAC}{\text{\textsc{la}}^{BB}}
\newcommand{\DBF}{\text{\textsc{dbf}}}

%================================================================================

%%sensitive part
\drop{\newcommand{\inti}{t^{int}_i}% \int is pre-defined
\newcommand{\inth}{\hat{t}^{int}_i}
\newcommand{\intj}{t^{int}_{i, l}}
\newcommand{\intjh}{\hat{t}^{int}_{i, l}}
\newcommand{\intjhp}[1]{\hat{t}^{int}_{i, #1}}
}
%% in this article i denote t^{int} by R or define R as t^{int}, in other articles 
%%\inti and so on should be defined like the dropped part above  
\newcommand{\inti}{R_i}% cause \int is pre-defined
\newcommand{\inth}{\widehat{R}_i}% 13-04 Chau hat-> widehat
\newcommand{\intj}{R_{i, l}}
\newcommand{\intjh}{\widehat{R}^{int}_{i, l}}% 13-04 Chau hat-> widehat
\newcommand{\intjhp}[1]{\widehat{R}_{i, #1}}% 13-04 Chau hat-> widehat

\def\R{R_i}
\newcommand{\Rp}[1]{{R_i}^{(#1)}}
%\newcommand{\Rub}{R^\mathsf{ub}}

\def\Rj{R_{i,l}}
\def\Rjh{\widehat{R}_{i,l}}
\newcommand{\Rjp}[1]{R_{i,#1}}
\newcommand{\Rjpp}[2]{R_{#1,#2}}
\newcommand{\Rjhp}[1]{\widehat{R}_{i,#1}}
\newcommand{\Rjhpp}[2]{\widehat{R}_{#1,#2}}

\def\N{N_{i}}
\def\Nh {\widehat{N}_i} % mark 13-4 Chau hat-> widehat


\def\Rh{\widehat{R}_{i,l}}
%%======================= Thuong modify ===============================
%\newcommand{\keywords}[1]{\par\addvspace\baselineskip
%\noindent\keywordname\enspace\ignorespaces#1}
%\newcommand{\whi}{\widehat{W}_{i}(t)}% already defined
%\newcommand{\wi}{W_{i}(t)}% defined
\newcommand{\lw}{\textit{last\_active}}
\newcommand{\rbfhj}{\widehat{\text{\textsc{rbf}}}(\tau_j, t)}
\newcommand{\rig}{\right\}}
\def\RTih{\widehat{RT}_{i}}
\def\RTl{\widehat{RT}_{i,l}}
\def\RTh{\widehat{RT}_{i,h}}
\def\I{{I}_{i}(t)}
\def\Ia{{I}_{i}(t_a)}
\def\Rhl{\widehat{R}_{i,l}}
\def\Rhh{\widehat{R}_{i,h}}
\def\cri{\hat{t}^*}
\def\Cs{C^s_i}
\def\Ts{T^s_i}
\def\Ds{D^s_i}
\def\Js{J^s_i}
\newcommand{\wc}{\textit{wcrt}} 
\def\tB{t^{bound}=\max_{j < i}\left\{(k-1)T_j - J_j\right\}}
\def\Rjs {R^s_{i,l}}
\def\RTjs {RT^s_{i,l}}
\def\RTs {RT^s_{i}}
\def\Ns {N^s_{i}}

%%======================= Manh modify ===============================

\newcommand{\Wm}[1]{{W_i}^{(#1)}}
\newcommand{\Wml}[1]{{W_{i,l}}^{(#1)}}
\newcommand{\Wmm}[2]{{W_{i,#1}}^{(#2)}}
\def\Cj{C_j}
\def\Uj{U_j}


%them 05-04-2013
\def\RT{RT_i}
\def\RTh{\widehat{RT}_{i}}
\def\RTj {RT_{i,l}}
\def\RTjh{\widehat{RT}_{i,l}}
\newcommand{\RTjp}[1]{RT_{i,#1}}
\newcommand{\RTjpp}[2]{RT_{#1,#2}}
\newcommand{\RTjhp}[1]{\widehat{RT}_{i,#1}}
\newcommand{\RTjhpp}[2]{\widehat{RT}_{#1,#2}}

\def\ac{last\_active_{a}}
\newcommand{\acp}[1]{last\_active_{#1}}
\newcommand{\whtp}[1]{\widehat{W}_{i, l}(#1)}
\newcommand{\wpt}[1]{{W}_{i, l}(#1)}

%=================================================================================
